\documentclass{beamer}

\mode<presentation>
{
  \usetheme{default}      % or try Darmstadt, Madrid, Warsaw, ...
  \usecolortheme{default} % or try albatross, beaver, crane, ...
  \usefonttheme{default}  % or try serif, structurebold, ...
  \setbeamertemplate{navigation symbols}{}
  \setbeamertemplate{footline}[frame number]
  \setbeamertemplate{caption}[numbered]
} 

\usepackage[english]{babel}
\usepackage[utf8x]{inputenc}

\usepackage{graphicx}
\usepackage{subcaption}
\usepackage{hyperref}

\title[Git for Economists]{Git and GitHub: A Guide for Economists}
\author{Frank Pinter}
\date{22 February 2019}

\AtBeginSection[]
{
  \begin{frame}
    \frametitle{Table of Contents}
    \tableofcontents[currentsection]
  \end{frame}
}

\begin{document}

\begin{frame}
  \titlepage
\end{frame}

\begin{frame}{Outline}
  \tableofcontents
\end{frame}

\section*{Introduction}

\begin{frame}{What is version control?}
Version control is a way to keep track of changes to code, text, and documents. And data and outputs.
\begin{itemize}
\item It gives you an organized revision history
\item It lets you experiment without fear
\item It lets you go back and forth between many different versions of the same file, and see a list of the differences
\item It makes (the technical aspects of) collaboration a breeze
\item It lets you and your collaborators work on different versions and then merge them
\end{itemize}
\end{frame}

\begin{frame}{What is Git?}
\begin{itemize}
\item Git is a program that does version control
\item It is the most popular version control program in software development
\item It is easy to set up and get started
\item There are many programs that add intuitive interfaces on top of Git
\item Git integrates seamlessly with online collaboration tools like GitHub and GitLab
\end{itemize}
\end{frame}

\section{The importance of version control}

\begin{frame}{Using it yourself}
Git isn't just useful for collaboration. It also helps you keep your own projects organized.
\end{frame}

\begin{frame}{Life without version control}
You're writing a paper and you have a regression.
\begin{itemize}
\item Advisor 1 tells you to include a certain variable. You put in lots of work to get the data, clean it, merge it, change the specification, and re-run.
\item Advisor 2 tells you that variable is dumb. You remove it.
\item Then Advisor 2 changes their mind.
\end{itemize}
What do you do?
\end{frame}

\begin{frame}{Life without version control}
Do you keep every specification you ever tried?
\begin{itemize}
\item The code and the outputs?
\item What if you discover a coding error that affects many of your specifications?
\item How do you organize all the files?
\end{itemize}
\end{frame}

\begin{frame}{Version control 0.1: putting dates in things}
Does this look familiar?
\texttt{run\_regs\_11\_17\_2018\_v4\_final\_final.do}

\vspace{1in}
``Not one piece of commercial software you have on your PC, your phone, your tablet,
your car, or any other modern computing device was written with the `date and initial' method.'' (Gentzkow and Shapiro)
\end{frame}

\begin{frame}{Version control 0.2: Dropbox}
\begin{itemize}
\item Dropbox keeps a crude version history.
\begin{itemize}
\item But there are no labels or comments, and it's not easy to see the differences between files.
\item So if you want to dig up ``the version where I had that other variable'' you have to manually look through a bunch of versions.
\item And good luck if you changed two scripts, not just one.
\end{itemize}
\item Dropbox lets you and your collaborators stay in sync.
\begin{itemize}
\item But what if you and your coauthor try to change the same script at the same time?
\item What if you are trying one change and, at the same time, your coauthor is trying a different change?
\end{itemize}
\end{itemize}
\end{frame}

\begin{frame}{Version control 0.2: Dropbox}
A Post It note spotted on a grad student's desk:
\begin{quote}
	Don't forget! At 10:18 am on November 17th, we changed the specification to add new variable.
\end{quote}
Don't live this way.
\end{frame}

\begin{frame}{Useful principles}
\begin{enumerate}
\item You're not going to remember why you did that.
\item Your coauthor can't read your mind.
\end{enumerate}

\end{frame}

\section{Using version control}

\begin{frame}{Why use Git?}
Git is the dominant version control system today. There are others, but they're generally more work with no benefit.
\end{frame}

\begin{frame}{What software can I use?}
\begin{itemize}
\item GitKraken
\item GitHub Desktop
\item Command line (powerful)
\item RStudio (for R projects)
\end{itemize}

\end{frame}

\section{Using Git by yourself}

\begin{frame}{Getting started}
\end{frame}

\begin{frame}{Commits: logging your changes}
\end{frame}

\begin{frame}{What should I include?}
\begin{enumerate}
\item At a minimum:
\begin{itemize}
\item Code (\texttt{.do}, \texttt{.R}, \texttt{.m}, \texttt{.jl}, and so on)
\item Text files (\texttt{.txt})
\item \LaTeX documents (\texttt{.tex})
\end{itemize}
\item I also recommend:
\begin{itemize}
\item Raw \texttt{.csv} datasets, if small (\textless 10 MB)
\end{itemize}
\item These are binary files, so you can't see differences between versions. I recommend including them anyway.
\begin{itemize}
\item PDF files
\item Word, Excel, PowerPoint files
\end{itemize}
\item Some people also include all datasets.
\begin{itemize}
\item Note that GitHub doesn't allow files larger than 100 MB, or projects with total size larger than 1 GB.
\end{itemize}
\end{enumerate}

For datasets, look into \href{https://git-lfs.github.com/}{Git Large File Storage}.
\end{frame}

\begin{frame}{Ignoring the junk}
In order to avoid driving your coauthors crazy, you \textbf{must} tell Git to ignore the junk files:
\begin{itemize}
\item Junk created by \LaTeX: \texttt{*.synctex.gz}, \texttt{*.out}, \texttt{*.log}, etc
\item Junk created by R: \texttt{.RData}
\item Junk created by Python: \texttt{*.pyc}
\end{itemize}
The way to do this is with a file called \texttt{.gitignore}
\end{frame}


\begin{frame}{Comparing commits: diff}
\end{frame}


\begin{frame}{Branches: trying things out}
\end{frame}

\begin{frame}{Keeping it local vs. using a remote repository}
Git doesn't require a remote repository. You can run it 100\% on your computer, with no connection to an outside server.
\begin{itemize}
\item This is useful if you have restrictions on your code (for example, you work with confidential health data)
\begin{itemize}
\item Ask me if you have questions about using Git this way on the NBER cluster
\end{itemize}
\item But a remote repository helps you keep things backed up seamlessly, and lets you collaborate
\end{itemize}

\end{frame}

\section{Using Git for collaboration}

\begin{frame}{Repos}
\end{frame}

\begin{frame}{What's GitHub?}
\end{frame}

\begin{frame}{Brief detour: hosting services}
\begin{itemize}
\item With a free GitHub account, you can create
\begin{itemize}
\item as many private repositories as you want
\item but each private repository can only have three collaborators.
\item You can also create as many public repositories as you want (and each can have as many collaborators as you want).
\end{itemize}
\item GitLab is a competing service. With a free GitLab account, you can create as many private repositories as you want, with as many collaborators as you want.
\item It's easy to use GitHub for one project, and GitLab for another
\end{itemize}

\end{frame}


\section*{Conclusion}

\begin{frame}{Conclusion}
\end{frame}

\begin{frame}{Further reading}
\begin{itemize}
\item Matt Gentzkow and Jesse Shapiro, ``Code and Data for the Social Sciences: A Practitioner’s Guide'' (\url{https://web.stanford.edu/~gentzkow/research/CodeAndData.pdf}). See Chapter 3 for more on why you should use version control.
\item Jes\'us Fern\'andez-Villaverde's notes on high-performance computing (see also his class Computational Economics). Chapter 5 (\url{https://www.sas.upenn.edu/~jesusfv/Chapter_HPC_5_Git.pdf}) is an extended Git tutorial using the command line interface.
\end{itemize}
\end{frame}

\end{document}
